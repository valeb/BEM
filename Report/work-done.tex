\chapter{Work Done}

\section{Implementation of the model}

The blade element momentum theory has been used to create a numerical model in python that takes a variety of parameter as input and returns the loads on the blades as well as the power coefficient of the turbine. For this, and iterative solution method has been chosen. In an initial guess it is assumed that the turbine does not effect the wind, therefore the linear and angular induction factors are zero, $a=0$ and $a'=0$. The problem is discretized by dividing the blade into $n$ elements of equal length $dr$. Next, for every blade element the angle of attack $\alpha$ is calculated using equation 3.1, where $\beta$ is the given local pitch angle and $\phi$ the local flow angle. The latter is given by equation 3.2 as function of the angular velocity $\Omega$, the incoming wind speed $U_0$, and the induction factors, and represents the angle of the incoming wind $U_{rel}$. The magnitude of $U_{rel}$ is given by equation 3.3. Using this angle of attack, the lift and drag coefficients are obtained from the curves of the used airfoil. Relations 3.4 and 3.5 are then used to calculate the thrust and torque on every individual blade element. By summing the contributions of each element, total thrust and toque are obtained from equations 3.6 and 3.7.\cite{manwell2010wind}

\begin{equation}\label{key}
\phi = atan \left( \frac{U_0(1-a)}{\Omega r (1+a')} \right)
\end{equation}

\begin{equation}\label{key}
\alpha = \phi - \beta
\end{equation}

\begin{equation}\label{key}
U_{rel} = \frac{U(1-a)}{sin(\phi)}
\end{equation}

\begin{equation}\label{key}
dT = N*\frac{1}{2}\rho U^2_{rel}(C_l cos(\phi)+C_d sin(\phi)c dr
\end{equation}

\begin{equation}\label{key}
dM = N*\frac{1}{2}\rho U^2_{rel}(C_l sin(\phi)-C_d cos(\phi)c r dr
\end{equation}

\begin{equation}\label{key}
M = \sum dM
\end{equation}

\begin{equation}\label{key}
T = \sum dT
\end{equation}

\subsection{<Sub-section title>}

\subsection{<Sub-section title>}
some text\cite{citation-2-name-here}, some more text
\subsection{<Sub-section title>}

\subsection{<Sub-section title>}

Refer figure \ref{fig:label}.

\begin{figure}[htb]
\centering
\includegraphics[scale=0.3]{./glider} % e.g. insert ./image for image.png in the working directory, adjust scale as necessary
\caption{<Caption here>}
\label{fig:label} % insert suitable label, this is used to refer to a fig from within the text as shown above
\end{figure}

\subsection{<Sub-section title>}


\section{<Section title>}

